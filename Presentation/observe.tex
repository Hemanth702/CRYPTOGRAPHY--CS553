\section{Observations}

\begin{frame}{ATTACKS}
    Proposed Complexity : $(D,T,M)=(2^{60},2^{66},2^{64})$\\
    Round key bits and their captured layers:
    \begin{itemize}
        \item 40-bit round key is captured in $18^{th}$ round key layer
        \item 12-bit round key is captured in $17^{th}$ round key layer
        \item 12-bit round key is captured in $1^{st}$ round key layer
    \end{itemize}
    This is erroneous because the differentials in the 1st and 17th rounds were unidentified viz $Y_1[10], Y_1[6], Y_1[2] and X_{17}[10], X_{17}[6], X_{17}[2]$.\\
    This leads to capturing only 58 bits in place of 64 as said, which make the time complexity $2^{70}$ by correcting $2^{66}$ as it needs exhaustive search.
\end{frame}

\begin{frame}{LINEAR LAYER}
\begin{block}{OBSERVATIONS AND IMPROVEMENTS}
\begin{enumerate}
    \item Improve hardware search,cover larger space
    \item Find more efficient constructions
    \item Explore trade-offs
    \item Extend to different platforms(PIC,ARM,etc,)
\end{enumerate}

\end{block}

\end{frame}
\begin{frame}{SECURITY}
\begin{block}{OBSERVATIONS AND IMPROVEMENTS}
\begin{enumerate}
    \item Zhao et al.: Differential analysis on Block Cipher PRIDE
    \item Found 16 different 2-round iterative characteristics
    \item constructed several 15-round differentials
    \item Based on these, launched differential attack on 18-round PRIDE
    \item Data,time,and memory complexity are $2^{60}$ $2^{66}$ $2^{64}$
    \item Even more security analysis.
\end{enumerate}

\end{block}

\end{frame}
