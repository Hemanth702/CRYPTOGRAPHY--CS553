\section{Brownie Point Nominations}
\begin{frame}{}
    \begin{block}{}
    The cipher PRIDE given is analyzed for different implementations and it is found out that PRIDE performs best in terms of Security when compared to SPECK and SIMON and other lightweight block ciphers. While SPECK and SIMON outperformed PRIDE in terms of efficiency, the security level of these when ranked will be in the order of:
		\begin{enumerate}
			\item PRIDE
			\item SPECK
			\item SIMON
		\end{enumerate}
	\end{block}
\end{frame}

\begin{frame}{}
\begin{block}{}
    \begin{itemize}
        \item The proposed complexity of 18-round differential attack $ (D,T,M) = (2^{60},2^{66},2^{64}) $ is again observed and it is found out that there are rounds where round key captures were said to be 64 in place of 58 bits. Re-evaluating the complexity we get $2^{70}$ in place of $(2^{66})$.\\
		So, the complexity now is $ (D,T,M) = (2^{60},2^{70},2^{64}) $
		
		\item Figure \ref{fig:1} and \ref{fig:2} are drawn using \texttt{$\backslash$tikzlibrary} package in latex with cryptographic symbols class.
		 \end{itemize}
		\end{block}
\end{frame}

